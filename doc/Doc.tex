\def\allfiles{}
\documentclass[11pt]{article}
\usepackage[a4paper]{geometry}
\geometry{left=2.0cm,right=2.0cm,top=2.5cm,bottom=2.5cm}
\usepackage{ctex} % 支持中文的LaTeX宏包
\usepackage{amsmath,amsfonts,graphicx,subfigure,amssymb,bm,amsthm,mathrsfs,mathtools,breqn} % 数学公式和符号的宏包集合
\usepackage{algorithm,algorithmicx} % 算法和伪代码的宏包
\usepackage[noend]{algpseudocode} % 算法和伪代码的宏包
\usepackage{fancyhdr} % 自定义页眉页脚的宏包
\usepackage[framemethod=TikZ]{mdframed} % 创建带边框的框架的宏包
\usepackage{fontspec} % 字体设置的宏包
\usepackage{adjustbox} % 调整盒子大小的宏包
\usepackage{fontsize} % 设置字体大小的宏包ddddddddd
\usepackage{tikz,xcolor} % 绘制图形和使用颜色的宏包
\usepackage{multicol} % 多栏排版的宏包
\usepackage{multirow} % 表格中合并单元格的宏包
\usepackage{pdfpages} % 插入PDF文件的宏包
\RequirePackage{listings} % 在文档中插入源代码的宏包
\RequirePackage{xcolor} % 定义和使用颜色的宏包
\usepackage{wrapfig} % 文字绕排图片的宏包
\usepackage{bigstrut,multirow,rotating} % 支持在表格中使用特殊命令的宏包
\usepackage{booktabs} % 创建美观的表格的宏包
\usepackage{circuitikz} % 绘制电路图的宏包
\usepackage{float} %
\usepackage{multirow}
\usepackage{indentfirst}
\usepackage{titlesec} %section宏
\usepackage{listings}
\usepackage{paralist}%压缩列表和行间列表环境
\usepackage[perpage]{footmisc}
\usepackage[colorlinks=false]{hyperref}
\usepackage{setspace}
%\usepackage[subfigure]{tocloft}      %必须这么写,否则会报错
%\renewcommand{\cftchapleader}{\cftdotfill{0.6}} %设置chapter条目的引导点间距
%\renewcommand{\cftsecleader}{\cftdotfill{0.6}}
%\renewcommand{\cftsubsecleader}{\cftdotfill{0.6}}
%\renewcommand{\cftchapfont}{\hts}    %设置chapter条目的字体
%\renewcommand{\cftsecfont}{\stxs}    %设置section条目的字体
%\renewcommand{\cftsubsecfont}{\stxs} %设置subsection条目的字体





\titleformat*{\section}{\LARGE\bfseries\songti}
\titleformat*{\subsection}{\Large\songti\bfseries}
\titleformat*{\subsubsection}{\large\songti\bfseries}
\titleformat*{\paragraph}{\large\songti\bfseries}
\titleformat*{\subparagraph}{\large\songti\bfseries}



\definecolor{dkgreen}{rgb}{0,0.6,0}
\definecolor{gray}{rgb}{0.5,0.5,0.5}
\definecolor{mauve}{rgb}{0.58,0,0.82}
\lstset{
	frame=tb,
	aboveskip=3mm,
	belowskip=3mm,
	showstringspaces=false,
	columns=flexible,
	framerule=1pt,
	rulecolor=\color{gray!35},
	%backgroundcolor=\color{gray\Delta 5},
	basicstyle={\small\ttfamily},
	numbers=none,
	numberstyle=\tiny\color{gray},
	keywordstyle=\color{blue},
	commentstyle=\color{dkgreen},
	stringstyle=\color{mauve},
	breaklines=true,
	breakatwhitespace=true,
	tabsize=3,
}


\lstdefinestyle{verilog}{
	columns=fixed,       
	numbers=left,                                        % 在左侧显示行号
	numberstyle=\tiny\color{gray},                       % 设定行号格式
	frame=single,                                          % 不显示背景边框
	backgroundcolor=\color[RGB]{255,255,255},            % 设定背景颜色
	keywordstyle=\color[RGB]{40,40,255},                 % 设定关键字颜色
	numberstyle=\footnotesize\color{darkgray},           
	commentstyle=\it\color[RGB]{0,96,96},                % 设置代码注释的格式
	stringstyle=\rmfamily\slshape\color[RGB]{128,0,0},   % 设置字符串格式
	showstringspaces=false,                              % 不显示字符串中的空格
	language=verilog,                                        % 设置语言
}

\lstdefinestyle{c}{
	columns=fixed,       
	numbers=left,                                        % 在左侧显示行号
	numberstyle=\tiny\color{gray},                       % 设定行号格式
	frame=single,                                          % 不显示背景边框
	backgroundcolor=\color[RGB]{255,255,255},            % 设定背景颜色
	keywordstyle=\color[RGB]{40,40,255},                 % 设定关键字颜色
	numberstyle=\footnotesize\color{darkgray},           
	commentstyle=\it\color[RGB]{0,96,96},                % 设置代码注释的格式
	stringstyle=\rmfamily\slshape\color[RGB]{128,0,0},   % 设置字符串格式
	showstringspaces=false,                              % 不显示字符串中的空格
	language=c,                                        % 设置语言
}




% 轻松引用, 可以用\cref{}指令直接引用, 自动加前缀. 
% 例: 图片label为fig:1
% \cref{fig:1} => Figure.1
% \ref{fig:1}  => 1
\usepackage[capitalize]{cleveref}
% \crefname{section}{Sec.}{Secs.}
\Crefname{section}{Section}{Sections}
\Crefname{table}{Table}{Tables}
\crefname{table}{Table.}{Tabs.}

%\setmainfont{Palatino Linotype.ttf}
%\setCJKmainfont{SimHei.ttf}
\setCJKsansfont{Songti.ttf}
% \setCJKmonofont{SimSun.ttf}
%\punctstyle{kaiming}
% 偏好的几个字体, 可以根据需要自行加入字体ttf文件并调用

\renewcommand{\emph}[1]{\begin{kaishu}#1\end{kaishu}}
\setlength{\parindent}{2em}
\everymath{\displaystyle}







\begin{document}
	\setlength{\baselineskip}{22pt}
	%若需在页眉部分加入内容, 可以在这里输入
	 \pagestyle{fancy}
	 \lhead{2023-2024学年秋季学期}
	 \chead{B09GB002Y-03\quad 程序设计基础与实验(C语言)}
	 \rhead{程序说明文档}
	\begin{center}
		\Huge 五子棋程序设计\\
		\Large 
	\end{center}
	\begin{center}
		\large \kaishu 姜俊彦\footnote{姜俊彦,中国科学院大学,2022K8009970011,jiangjunyan22@mails.ucas.ac.cn}
	\end{center}
	\begin{spacing}{1.5}
		\tableofcontents
	\end{spacing}
	\setcounter{page}{1}
	\thispagestyle{fancy}
	\newpage
	\section{实验题目}
	\large\kaishu  五子棋程序设计:要求实现包含以下功能的五子棋程序
	
	\begin{compactitem}
		\item 人人对战
		\begin{compactitem}
			\item 棋盘与落子显示
			\item 键盘输入落子与异常输入检测
			\item 模式选择与退出
			\item 裁判与禁手
		\end{compactitem}
		\item 人机对战
		\begin{compactitem}
			\item 显示人机落子
			\item 五子棋程序比赛
		\end{compactitem}
	\end{compactitem}
	\section{实验概况}
	本实验为来到中国科学大学,就读于网络空间安全专业的第三个系统性实验(前两次分别为:计算机科学技术导论动态网页设计与数字电路AX7035 FPGA开发板电路设计),也是可以说是对于C语言学习的总结与收官。本实验自2023年12月9日项目创立至今(2024年1月20日),共43日。本人于此期间基本完成了实验所要求的的全部内容。接下来是对程序本身及程序编写过程中的收获与反思。
	\section{实验过程}
	\songti
	\subsection{实验平台}
	\begin{compactitem}
		\item 操作系统:Windows 11 专业版 22H2, Ubuntu 22.04(64-bit)
		\item 编辑平台:Visual Studio Code 1.85.2
		\item 编译环境:gcc version 11.4.0
		\item 代码管理平台:GitHub
	\end{compactitem}
	
	\subsection{实验程序}
	限于篇幅这里不再粘贴大段源码,有意者可于日后前往我的GitHub仓库查看\footnote{https://github.com/jiuhao47}
	\subsubsection{整体架构}
	本程序大致按照以下几个主要模块(文件)架构组成。
	\begin{compactitem}
		\item 棋盘、落子显示-chessBoard.c
		\item 常用函数-functions.c
		\item 输入信息接受与处理-inputprocess.c
		\item 状态控制与信息显示-state.c
		\item 裁判(含胜负及禁手)-judge.c
		\item 评分函数-judge\_score.c
		\item 树架构AI-ai.c
		\item 主函数及头文件-main.c\&myhead.h
	\end{compactitem}
	
	接下来将对每个部分分块介绍主要功能及其关联
	\subsubsection{main.c}
	鉴于是最短的文件,就将其整个粘贴进来了。这样做的思路来源其实是从《Python编程从入门到实践》这本书中获得的\footnote{本人于2023年暑假期间对照此书学习了一点Python编程的知识,其中有一个章节介绍了游戏编程过程中的面向对象思想,而正是此影响着我的五子棋程序编写}
	\begin{lstlisting}[style=c]
	#include "head.h"
	int main()
	{
		init_state();
		input();
		update();
		while (gamestates.runningstate)
		{
			update();
		}
		return 0;
	}
	\end{lstlisting}
	
	除必要的的初始化外,程序全部由gamestates结构体中的成员runningstate控制程序运行走向。而具体的程序内容则被封装在update()函数中,下为update()\footnote{位于chessboard.c中}函数的片段,其主体是以状态机为核心搭建的,不同的状态导向不同的控制流。
	\begin{lstlisting}[style=c]
	//...
	if (gamestates.runningstate == -2)
	{
		gamestates.runningstate = 0;
		return;
	}
	else if (gamestates.runningstate == 1)
	{
		updateHumanInput();
	}
	//...
	\end{lstlisting}
	\subsubsection{chessboard.c}
	棋盘、落子显示位于此模块内。主要是利用双循环进行复位、更新、打印等。此过程中更复杂的其实是对于中文编码\footnote{GBK与UTF-8的占字节大小不同}的理解
	\begin{lstlisting}[style=c]
	for (i = 0; i < SIZE; i++)
	{
		for (j = 0; j < SIZE; j++)
		{
			...//大量的双循环结构
		}
	}
	\end{lstlisting}
	\subsubsection{function.c}
	一些常用的函数包,比如读取一行内容、坐标转换、乘方、最值、绝对值等
	\begin{compactitem}
		\item int mygetline(char s[],int lim)
		\item int pointInBoard(int tempx,int tempy)
		\item int displayPosToInnerPos(int tempx,int tempy)
		\item int mypow(int x, int n)
		\item int weight(int x,int y)
		\item int mymax(int x,int y)
		\item int myabs(int x)
	\end{compactitem}
	\subsubsection{inputprocess.c}
	该模块重在对于输入内容的处理,这里需要考虑到不同情况的输入(坐标还是模式)。由于事先设计了状态机,故此处也使用了状态机的错误状态来处理非法输入(ErrorHandle()函数)。无太多亮点不多赘述。
	\subsubsection{state.c}
	对于整个五子棋程序可能相对设计的比较精巧的就是state.c这个模块,gamestates为全局变量。所有函数可见,其作为调控整个程序运行的状态机控制程序的运行逻辑
	\begin{compactitem}
		\item 
		\item 
		\item 
		
	\end{compactitem}
	
\end{document}


