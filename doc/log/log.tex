%\documentclass[11pt]{article}
\usepackage[a4paper]{geometry}
\geometry{left=2.0cm,right=2.0cm,top=2.5cm,bottom=2.5cm}
\usepackage{ctex} % 支持中文的LaTeX宏包
\usepackage{amsmath,amsfonts,graphicx,subfigure,amssymb,bm,amsthm,mathrsfs,mathtools,breqn} % 数学公式和符号的宏包集合
\usepackage{algorithm,algorithmicx} % 算法和伪代码的宏包
\usepackage[noend]{algpseudocode} % 算法和伪代码的宏包
\usepackage{fancyhdr} % 自定义页眉页脚的宏包
\usepackage[framemethod=TikZ]{mdframed} % 创建带边框的框架的宏包
\usepackage{fontspec} % 字体设置的宏包
\usepackage{adjustbox} % 调整盒子大小的宏包
\usepackage{fontsize} % 设置字体大小的宏包ddddddddd
\usepackage{tikz,xcolor} % 绘制图形和使用颜色的宏包
\usepackage{multicol} % 多栏排版的宏包
\usepackage{multirow} % 表格中合并单元格的宏包
\usepackage{pdfpages} % 插入PDF文件的宏包
\RequirePackage{listings} % 在文档中插入源代码的宏包
\RequirePackage{xcolor} % 定义和使用颜色的宏包
\usepackage{wrapfig} % 文字绕排图片的宏包
\usepackage{bigstrut,multirow,rotating} % 支持在表格中使用特殊命令的宏包
\usepackage{booktabs} % 创建美观的表格的宏包
\usepackage{circuitikz} % 绘制电路图的宏包
\usepackage{float} %
\usepackage{multirow}
\usepackage{indentfirst}
\usepackage{titlesec} %section宏
\usepackage{listings}
\usepackage{paralist}%压缩列表和行间列表环境
\usepackage[perpage]{footmisc}
\usepackage[colorlinks=false]{hyperref}
\usepackage{setspace}
%\usepackage[subfigure]{tocloft}      %必须这么写,否则会报错
%\renewcommand{\cftchapleader}{\cftdotfill{0.6}} %设置chapter条目的引导点间距
%\renewcommand{\cftsecleader}{\cftdotfill{0.6}}
%\renewcommand{\cftsubsecleader}{\cftdotfill{0.6}}
%\renewcommand{\cftchapfont}{\hts}    %设置chapter条目的字体
%\renewcommand{\cftsecfont}{\stxs}    %设置section条目的字体
%\renewcommand{\cftsubsecfont}{\stxs} %设置subsection条目的字体





\titleformat*{\section}{\LARGE\bfseries\songti}
\titleformat*{\subsection}{\Large\songti\bfseries}
\titleformat*{\subsubsection}{\large\songti\bfseries}
\titleformat*{\paragraph}{\large\songti\bfseries}
\titleformat*{\subparagraph}{\large\songti\bfseries}



\definecolor{dkgreen}{rgb}{0,0.6,0}
\definecolor{gray}{rgb}{0.5,0.5,0.5}
\definecolor{mauve}{rgb}{0.58,0,0.82}
\lstset{
	frame=tb,
	aboveskip=3mm,
	belowskip=3mm,
	showstringspaces=false,
	columns=flexible,
	framerule=1pt,
	rulecolor=\color{gray!35},
	%backgroundcolor=\color{gray\Delta 5},
	basicstyle={\small\ttfamily},
	numbers=none,
	numberstyle=\tiny\color{gray},
	keywordstyle=\color{blue},
	commentstyle=\color{dkgreen},
	stringstyle=\color{mauve},
	breaklines=true,
	breakatwhitespace=true,
	tabsize=3,
}


\lstdefinestyle{verilog}{
	columns=fixed,       
	numbers=left,                                        % 在左侧显示行号
	numberstyle=\tiny\color{gray},                       % 设定行号格式
	frame=single,                                          % 不显示背景边框
	backgroundcolor=\color[RGB]{255,255,255},            % 设定背景颜色
	keywordstyle=\color[RGB]{40,40,255},                 % 设定关键字颜色
	numberstyle=\footnotesize\color{darkgray},           
	commentstyle=\it\color[RGB]{0,96,96},                % 设置代码注释的格式
	stringstyle=\rmfamily\slshape\color[RGB]{128,0,0},   % 设置字符串格式
	showstringspaces=false,                              % 不显示字符串中的空格
	language=verilog,                                        % 设置语言
}

\lstdefinestyle{c}{
	columns=fixed,       
	numbers=left,                                        % 在左侧显示行号
	numberstyle=\tiny\color{gray},                       % 设定行号格式
	frame=single,                                          % 不显示背景边框
	backgroundcolor=\color[RGB]{255,255,255},            % 设定背景颜色
	keywordstyle=\color[RGB]{40,40,255},                 % 设定关键字颜色
	numberstyle=\footnotesize\color{darkgray},           
	commentstyle=\it\color[RGB]{0,96,96},                % 设置代码注释的格式
	stringstyle=\rmfamily\slshape\color[RGB]{128,0,0},   % 设置字符串格式
	showstringspaces=false,                              % 不显示字符串中的空格
	language=c,                                        % 设置语言
}




% 轻松引用, 可以用\cref{}指令直接引用, 自动加前缀. 
% 例: 图片label为fig:1
% \cref{fig:1} => Figure.1
% \ref{fig:1}  => 1
\usepackage[capitalize]{cleveref}
% \crefname{section}{Sec.}{Secs.}
\Crefname{section}{Section}{Sections}
\Crefname{table}{Table}{Tables}
\crefname{table}{Table.}{Tabs.}

%\setmainfont{Palatino Linotype.ttf}
%\setCJKmainfont{SimHei.ttf}
\setCJKsansfont{Songti.ttf}
% \setCJKmonofont{SimSun.ttf}
%\punctstyle{kaiming}
% 偏好的几个字体, 可以根据需要自行加入字体ttf文件并调用

\renewcommand{\emph}[1]{\begin{kaishu}#1\end{kaishu}}
\setlength{\parindent}{2em}
\everymath{\displaystyle}








\ifx\allfiles\undefined
\begin{document}
\else
\fi


\begin{compactitem}
	\item 2023-11-23:
	\begin{compactitem}
		\item 建立Github协作库、添加模板文件top.sv
		\item 拿到AX7035 FPGA 开发板
	\end{compactitem}
	\item 2023-11-24:
	\begin{compactitem}
		\item Github协作邀请完毕
	\end{compactitem}
	\item 2023-12-07:
	\begin{compactitem}
		\item 总体布局与任务分发
		\begin{compactitem}
			\item Edgedetect.v-刘镇豪
			\item KillShake.v-刘镇豪
			\item FIFO.v-吴尚哲
			\item LED\_display.v-吴尚哲
			\item binary\_20b\_to\_bcd\_6d.v-规划中
		\end{compactitem}
		\item SRAM片上内存
		\begin{compactitem}
			\item 完成了.xdc管脚协议的补充(后证实不需要)
			\item 研究了SRAM的结构与原理
		\end{compactitem}
		\item 库文件细化
		\begin{compactitem}
			\item 建立了参考文献集Reference.txt
			\item 建立了重要信息共享文档ShareLog.md
			\item 建立了样本数据集datadic.txt
		\end{compactitem}
\end{compactitem}
	\item 2023-12-09 
	\begin{compactitem}
		\item 关于AX7035开发板
		\begin{compactitem}
			\item 找到了一份完备的教程
		\end{compactitem}
		\item 关于DDR3
		\begin{compactitem}
			\item 建立了DDR3的功能及驱动模块
			\item 建立了mem\_burst.v的读写模块,但是还未来得及分析
		\end{compactitem}
		\item 关于.xdc文件
		\begin{compactitem}
			\item 恢复了原.xdc样式,并对修改做了备份
		\end{compactitem}
		\item 关于top.sv
		\begin{compactitem}
			\item 仿照样例撰写了led7seg\_decode.v,本质为0-9二进制数到8端数码管数据译码器
			\item 写了一些注释:其中下面一段代码存疑
			\begin{lstlisting}[style=verilog]
genvar i;
generate 
	for(i=0; i<6; i=i+1) begin
		led7seg_decode d(cnt[i*4 +: 4], 1'b1, seg[i*8 +: 8]);//+是做什么的?
end
endgenerate
			\end{lstlisting}
		\end{compactitem}
		\item 关于组员
		\begin{compactitem}
			\item FIFO.v已完成
			\item LED\_display.v已完成
			\item KillShake.v已完成
			\item Edgedetect.v已完成
		\end{compactitem}
	\end{compactitem}
	\item 2023-12-17
	\begin{compactitem}
		\item 关于top.sv
		\begin{compactitem}
			\item 实现了防抖电路和脉冲输出的测试
			\item 撰写了指示灯显示与状态切换代码
		\end{compactitem}
	\end{compactitem}
	\item 2023-12-18
	\begin{compactitem}
		\item 关于top.sv
		\begin{compactitem}
			\item 实现了按键与`LED`灯对应的代码与测试-刘镇豪
			\item 探索了筛法的可能性并决定算法为埃氏筛法,初步完成了埃氏筛法的代码实现,未测试
		\end{compactitem}
		\item 总体任务分发
		\begin{compactitem}
			\item binary\_20b\_to\_bcd\_6d.v-吴尚哲
			\item Count\_to\_one\_second.v-刘镇豪
		\end{compactitem}
		\item 关于组员完成情况
		\begin{compactitem}
			\item Count\_to\_one\_second.v已完成,未测试
		\end{compactitem}
	\end{compactitem}
	\item 2023-12-23
	\begin{compactitem}
		\item 关于top.sv
		\begin{compactitem}
			\item 实现了八段数码管显示的代码编写及测试(10进制)
			\item 实现了一秒计时器的整合与编写
			\item 实现了埃氏筛法(算法层面),但是其对于内存地址的调用目前仍然存在问题
		\end{compactitem}
		\item 关于组员
		\begin{compactitem}
			\item binary\_20b\_to\_bcd\_6d.v已完成,已测试
			\item Count\_to\_one\_second.v已测试
		\end{compactitem}
	\end{compactitem}
	\item 2023-12-24
	\begin{compactitem}
		\item 关于top.sv
		\begin{compactitem}
			\item 实现了埃氏筛法,最快输出达到1s之内完成
		\end{compactitem}
	\end{compactitem}
	\item 2023-12-25
	\begin{compactitem}
		\item 关于top.sv
		\begin{compactitem}
			\item 实现了最快输出的递增和递减功能按钮对应,但是对于1s输出的复位目前仍存在问题
		\end{compactitem}
		\item 关于组员
		\begin{compactitem}
			\item 布置了实验报告撰写的相关任务
		\end{compactitem}
	\end{compactitem}
	\item 2023-12-26
	\begin{compactitem}
		\item 关于top.sv
		\begin{compactitem}
			\item 实现了实验要求的所有功能
			\item 美化了整体代码布局
		\end{compactitem}
		\item 关于库文件
		\begin{compactitem}
			\item 将所有模块分装为.v文件存储在src文件夹下
			\item 将未用到的代码及内容存储在misc/Unused文件夹下
		\end{compactitem}
	\end{compactitem}
	\item 2023-12-28
	\begin{compactitem}
		\item 关于实验报告
		\begin{compactitem}
			\item 完成了实验报告撰写
		\end{compactitem}
		\item 关于算法性能
		\begin{compactitem}
			\item 通过统计时钟周期数估计了算法性能
		\end{compactitem}
	\end{compactitem}
\end{compactitem}






\ifx\allfiles\undefined
\end{document}
\else
\fi

